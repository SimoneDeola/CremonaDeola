\documentclass[../../codeInspection.tex]{subfiles}

\begin{document}

	\chapter{Functional Role of the Classes Assigned}

	Here will be given the functional role of the StandardService class and the methods assigned to us.

	\section{The Class}
		StandardService is the implementation of the "Service" interface that represents a set of Connectors that share a single Container.\\
		A Connector is in charge to receive requests from clients and return responses to them.
		A Connector follows this logic:
		\begin{enumerate}
				\item Receive a request.
				\item Create a Request and Response instance and populate it with the right properties.
				\item Identify the right Container (in this case the one shared with the Service) to process the Request.
				\item Make the Container process the Request.
				\item Return to the client the result obtained from the Container.
		\end{enumerate}

		A Container is in charge to process requests received from clients and return responses.

	\section{Method: addConnector}

		This method is in charge to add a new Connector to the set of Connectors of the current Service.

		\subsection{Javadoc}
			\lstinputlisting[language=Java, firstline=380, lastline=385, firstnumber=380]{../src/StandardService.java}

		\subsection{Functioning}
			All the body of this method is in a Synchronized block, so it's thread-safe.\\
			The Connector that have to be added is initialized with the Container and the current Service:
			\lstinputlisting[language=Java, firstline=389, lastline=390, firstnumber=389]{../src/StandardService.java}

			The Connector is actually added to the set of Connectors:
			\lstinputlisting[language=Java, firstline=391, lastline=394, firstnumber=391]{../src/StandardService.java}

			The Connector is initialized if the Service is already initialized:
			\lstinputlisting[language=Java, firstline=396, lastline=402, firstnumber=396]{../src/StandardService.java}

			If the Connector implements the Lifecycle interface and the current Service is already started, then the start() method of the Connector is called making him run:
			\lstinputlisting[language=Java, firstline=404, lastline=410, firstnumber=404]{../src/StandardService.java}

			All the listeners on the Connector are notified of the changes:
			\lstinputlisting[language=Java, firstline=413, lastline=413, firstnumber=413]{../src/StandardService.java}

	\section{Method: removeConnector}

		This method is in charge to remove a new Connector to the set of Connectors of the current Service.

		\subsection{Javadoc}
			\lstinputlisting[language=Java, firstline=467, lastline=473, firstnumber=467]{../src/StandardService.java}

		\subsection{Functioning}
			All the body of this method is in a Synchronized block, so it's thread-safe.\\

			The index of the Connector that has to be removed is found:
			\lstinputlisting[language=Java, firstline=480, lastline=486, firstnumber=480]{../src/StandardService.java}

			If the Connector isn't in the set of Connectors associated to this Service, then the method breaks:
			\lstinputlisting[language=Java, firstline=488, lastline=489, firstnumber=488]{../src/StandardService.java}

			The interested Connector is stopped:
			\lstinputlisting[language=Java, firstline=499, lastline=499, firstnumber=499]{../src/StandardService.java}

			The Connector is actually removed:
			\lstinputlisting[language=Java, firstline=506, lastline=512, firstnumber=506]{../src/StandardService.java}

			All the listeners on the Connector are notified of the changes:
			\lstinputlisting[language=Java, firstline=515, lastline=515, firstnumber=515]{../src/StandardService.java}

	\section{Method: start}

		This method is in charge to initialize and start all the Connectors and the Container associated to this Service.

		\subsection{Javadoc}
			\lstinputlisting[language=Java, firstline=578, lastline=586, firstnumber=578]{../src/StandardService.java}

		\subsection{Functioning}

			If the Service is already started, then the information is logged:
			\lstinputlisting[language=Java, firstline=590, lastline=594, firstnumber=590]{../src/StandardService.java}

			If the Service is not already initialized, then it will be initialized:
			\lstinputlisting[language=Java, firstline=596, lastline=597, firstnumber=596]{../src/StandardService.java}

			The BEFORE\textunderscore START\textunderscore EVENT, STARTING\textunderscore SERVICE and START\textunderscore EVENT are logged to the listeners and the "started" flag is set as true:
			\lstinputlisting[language=Java, firstline=600, lastline=606, firstnumber=600]{../src/StandardService.java}

			Connectors and the Container are started:
			\lstinputlisting[language=Java, firstline=609, lastline=615, firstnumber=609]{../src/StandardService.java}
			\lstinputlisting[language=Java, firstline=618, lastline=623, firstnumber=618]{../src/StandardService.java}

			All the listeners are notified of the changes:
			\lstinputlisting[language=Java, firstline=626, lastline=626, firstnumber=626]{../src/StandardService.java}

	\section{Method: stop}

		This method is in charge to stop every Connector and the Container associated to this Service. It is also in charge to avoid the public methods to be called from the outside.

		\subsection{Javadoc}
			\lstinputlisting[language=Java, firstline=631, lastline=639, firstnumber=631]{../src/StandardService.java}

		\subsection{Functioning}

			If the Service is not already started then it cannot be stopped:
			\lstinputlisting[language=Java, firstline=643, lastline=645, firstnumber=643]{../src/StandardService.java}

			The BEFORE\textunderscore STOP\textunderscore EVENT, STOP\textunderscore EVENT and STOPPING\textunderscore SERVICE are logged to the listeners and the "started" flag is set as false:
			\lstinputlisting[language=Java, firstline=648, lastline=655, firstnumber=648]{../src/StandardService.java}

			Connectors and Container are stopped:
			\lstinputlisting[language=Java, firstline=658, lastline=663, firstnumber=658]{../src/StandardService.java}
			\lstinputlisting[language=Java, firstline=666, lastline=672, firstnumber=666]{../src/StandardService.java}

			All the listeners are notified of the changes:
			\lstinputlisting[language=Java, firstline=675, lastline=675, firstnumber=675]{../src/StandardService.java}

	\section{Method: initialize}

		This method is in charge to initialize the Service for the active use.

		\subsection{Javadoc}
			\lstinputlisting[language=Java, firstline=680, lastline=683, firstnumber=680]{../src/StandardService.java}

		\subsection{Functioning}

			If the Service is already initialized then the SERVICE\textunderscore HAS\textunderscore BEEN\textunderscore INIT message is notified to the listeners and the method will return; otherwise, the "initialized" flag is set as true:
			\lstinputlisting[language=Java, firstline=688, lastline=694, firstnumber=688]{../src/StandardService.java}

			The Object Name is initialized and the Service is registered in the current Server:
			\lstinputlisting[language=Java, firstline=696, lastline=714, firstnumber=696]{../src/StandardService.java}

			Associated Connectors are initialized:
			\lstinputlisting[language=Java, firstline=718, lastline=723, firstnumber=718]{../src/StandardService.java}



\end{document}