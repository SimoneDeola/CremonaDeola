\documentclass[../../codeInspection.tex]{subfiles}

\begin{document}

	\chapter{Additional Material}

		\section{Code Inspection Checklist}

			This is the checklist used for the code inspection:

			\begin{itemize}
				
				\item  	\textbf{Naming Conventions}
						\begin{itemize}

					  		\item C.01)\label{C:01} All class names, interface names, method names, class variables, method variables, and constants used should have meaningful names and do what the name suggests.

					  		\item C.02)\label{C:02} If one-character variables are used, they are used only for temporary “throwaway” variables, such as those used in for loops.

					  		\item C.03)\label{C:03} Class names are nouns, in mixed case, with the first letter of each word in capitalized. Examples: class Raster; class ImageSprite;

					  		\item C.04)\label{C:04} Interface names should be capitalized like classes.

					  		\item C.05)\label{C:05} Method names should be verbs, with the first letter of each addition word capitalized. Examples: getBackground(); computeTemperature().

					  		\item C.06)\label{C:06} Class variables, also called attributes, are mixed case, but might begin with an underscore (‘\textunderscore’) followed by a lowercase first letter. All the remaining words in the variable name have their first letter capitalized. Examples: \textunderscore windowHeight, timeSeriesData.

					  		\item C.07)\label{C:07} Constants are declared using all uppercase with words separated by an underscore. Examples: MIN\textunderscore WIDTH; MAX\textunderscore HEIGHT;

					  	\end{itemize}

				\item 	\textbf{Indention}
						\begin{itemize}

							\item C.08)\label{C:08} Three or four spaces are used for indentation and done so consistently.

							\item C.09)\label{C:09} No tabs are used to indent.

						\end{itemize}

				\item 	\textbf{Braces}
						\begin{itemize}

							\item C.10)\label{C:10} Consistent bracing style is used, either the preferred “Allman” style (first brace goes underneath the opening block) or the “Kernighan and Ritchie” style (first brace is on the same line of the instruction that opens the new block).

							\item C.11)\label{C:11} All if, while, do-while, try-catch, and for statements that have only one statement to execute are surrounded by curly braces. Example:
							Avoid this:
							\begin{lstlisting}[language=Java]
							if ( condition )
								doThis();
							\end{lstlisting}
							Instead do this:
							\begin{lstlisting}[language=Java]
							if ( condition ) {
								doThis(); 
							}
							\end{lstlisting}

						\end{itemize}

				\item 	\textbf{File Organization}
						\begin{itemize}

							\item C.12)\label{C:12} Blank lines and optional comments are used to separate sections (beginning comments, package/import statements, class/interface declarations which include class variable/attributes declarations, constructors, and methods).

							\item C.13)\label{C:13} Where practical, line length does not exceed 80 characters.

							\item C.14)\label{C:14} When line length must exceed 80 characters, it does NOT exceed 120 characters.

						\end{itemize}

				\item 	\textbf{Wrapping Lines}
						\begin{itemize}

							\item C.15)\label{C:15} Line break occurs after a comma or an operator.

							\item C.16)\label{C:16} Higher-level breaks are used.

							\item C.17)\label{C:17} A new statement is aligned with the beginning of the expression at the same level as the previous line.

						\end{itemize}

				\item 	\textbf{Comments}
						\begin{itemize}

							\item C.18)\label{C:18} Comments are used to adequately explain what the class, interface, methods, and blocks of code are doing.

							\item C.19)\label{C:19} Commented out code contains a reason for being commented out and a date it can be removed from the source file if determined it is no longer needed.

						\end{itemize}
				
				\item 	\textbf{Java Source Files}
						\begin{itemize}

							\item C.20)\label{C:20} Each Java source file contains a single public class or interface.

							\item C.21)\label{C:21} The public class is the first class or interface in the file.

							\item C.22)\label{C:22} Check that the external program interfaces are implemented consistently with what is described in the javadoc.

							\item C.23)\label{C:23} Check that the javadoc is complete (i.e., it covers all classes and files part of the set of classes assigned to you).

						\end{itemize}

				\item 	\textbf{Package and Import Statements}
						\begin{itemize}

							\item C.24)\label{C:24} If any package statements are needed, they should be the first non- comment statements. Import statements follow.

						\end{itemize}

				\item 	\textbf{Class and Interface Declarations}
						\begin{itemize}

							\item C.25)\label{C:25} The class or interface declarations shall be in the following order:
								\begin{enumerate}

									\item Class/interface documentation comment.

									\item Class or interface statement.

									\item Class/interface implementation comment, if necessary.

									\item Class (static) variables.

										\begin{enumerate}

											\item First public class variables.

											\item Next protected class variables.

											\item Next package level (no access modifier).

											\item Last private class variables.

										\end{enumerate}

									\item Instance variables.

										\begin{enumerate}

											\item First public instance variables.

											\item Next protected instance variables.

											\item Next package level (no access modifier).

											\item Last private instance variables.

										\end{enumerate}

									\item Constructors.

									\item Methods.

								\end{enumerate}

							\item C.26)\label{C:26} Methods are grouped by functionality rather than by scope or accessibility.

							\item C.27)\label{C:27} Check that the code is free of duplicates, long methods, big classes, breaking encapsulation, as well as if coupling and cohesion are adequate.

						\end{itemize}

				\item 	\textbf{Initialization and Declarations}
						\begin{itemize}

							\item C.28)\label{C:28} Check that variables and class members are of the correct type. Check that they have the right visibility (public/private/protected).

							\item C.29)\label{C:29} Check that variables are declared in the proper scope.

							\item C.30)\label{C:30} Check that constructors are called when a new object is desired.

							\item C.31)\label{C:31} Check that all object references are initialized before use.

							\item C.32)\label{C:32} Variables are initialized where they are declared, unless dependent upon a computation.

							\item C.33)\label{C:33} Declarations appear at the beginning of blocks (A block is any code surrounded by curly braces “{“ and “}” ). The exception is a variable can be declared in a ‘for’ loop.

						\end{itemize}

				\item 	\textbf{Method Calls}
						\begin{itemize}

							\item C.34)\label{C:34} Check that parameters are presented in the correct order.

							\item C.35)\label{C:35} Check that the correct method is being called, or should it be a different method with a similar name.

							\item C.36)\label{C:36} Check that method returned values are used properly.

						\end{itemize}

				\item 	\textbf{Arrays}
						\begin{itemize}

							\item C.37)\label{C:37} Check that there are no off-by-one errors in array indexing (that is, all required array elements are correctly accessed through the index).

							\item C.38)\label{C:38} Check that all array (or other collection) indexes have been prevented from going out-of-bounds.

							\item C.39)\label{C:39} Check that constructors are called when a new array item is desired.

						\end{itemize}

				\item 	\textbf{Object Comparison}
						\begin{itemize}

							\item C.40)\label{C:40} Check that all objects (including Strings) are compared with "equals" and not with "==".

						\end{itemize}

				\item 	\textbf{Output Format}
						\begin{itemize}

							\item C.41)\label{C:41} Check that displayed output is free of spelling and grammatical errors.

							\item C.42)\label{C:42} Check that error messages are comprehensive and provide guidance as to how to correct the problem.

							\item C.43)\label{C:43} Check that the output is formatted correctly in terms of line stepping and spacing.

						\end{itemize}

				\item 	\textbf{Computation, Comparisons and Assignments}
						\begin{itemize}

							\item C.44)\label{C:44} Check that the implementation avoids “brutish programming: (see \url{http://users.csc.calpoly.edu/~jdalbey/SWE/CodeSmells/bonehead.html}).

							\item C.45)\label{C:45} Check order of computation/evaluation, operator precedence and parenthesizing.

							\item C.46)\label{C:46} Check the liberal use of parenthesis is used to avoid operator precedence problems.

							\item C.47)\label{C:47} Check that all denominators of a division are prevented from being zero.

							\item C.48)\label{C:48} Check that integer arithmetic, especially division, are used appropriately to avoid causing unexpected truncation/rounding.

							\item C.49)\label{C:49} Check that the comparison and Boolean operators are correct.

							\item C.50)\label{C:50} Check throw-catch expressions, and check that the error condition is actually legitimate.

							\item C.51)\label{C:51} Check that the code is free of any implicit type conversions.

						\end{itemize}

				\item 	\textbf{Exceptions}
						\begin{itemize}

							\item C.52)\label{C:52} Check that the relevant exceptions are caught.

							\item C.53)\label{C:53} Check that the appropriate action are taken for each catch block.

						\end{itemize}

				\item 	\textbf{Flow of Control}
						\begin{itemize}

							\item C.54)\label{C:54} In a switch statement, check that all cases are addressed by break or return.

							\item C.55)\label{C:55} Check that all switch statements have a default branch.

							\item C.56)\label{C:56} Check that all loops are correctly formed, with the appropriate initialization, increment and termination expressions.

						\end{itemize}

				\item 	\textbf{Files}
						\begin{itemize}

							\item C.57)\label{C:57} Check that all files are properly declared and opened.

							\item C.58)\label{C:58} Check that all files are closed properly, even in the case of an error.

							\item C.59)\label{C:59} Check that EOF conditions are detected and handled correctly.

							\item C.60)\label{C:60} Check that all file exceptions are caught and dealt with accordingly.
							
						\end{itemize}
			\end{itemize}

\end{document}