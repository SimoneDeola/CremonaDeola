\documentclass[../../../../codeInspection.tex]{subfiles}

\begin{document}

	\section{Method: removeConnector}

		On this section are spotted all the issues founded on the method removeConnector following the assigned checklist.
		The method removeConnector is located from the line 467 to the line 518 on the StandardService class.

		\begin{itemize}

			\item \textbf{Checklist \hyperref[C:02]{C.02}} \\

				  In this method it's declared "j", an integer variable that is not used for temporary use. Replace with a significative name.

			\item \textbf{Checklist \hyperref[C:11]{C.11}} \\

				  \lstinputlisting[language=Java, firstline=488, lastline=489, firstnumber=488]{../src/StandardService.java}

				  \lstinputlisting[language=Java, firstline=509, lastline=510, firstnumber=509]{../src/StandardService.java}

				  These lines of code contains if statements with only one statement to execute and are not surrounded by curly braces.

		    \item \textbf{Checklist \hyperref[C:18]{C.18}} \\

		    	  \lstinputlisting[language=Java, firstline=474, lastline=474, firstnumber=474]{../src/StandardService.java}

		    	  \lstinputlisting[language=Java, firstline=477, lastline=477, firstnumber=477]{../src/StandardService.java}

		    	  \lstinputlisting[language=Java, firstline=491, lastline=491, firstnumber=491]{../src/StandardService.java}

		    	  \lstinputlisting[language=Java, firstline=500, lastline=500, firstnumber=500]{../src/StandardService.java}

		    	  \lstinputlisting[language=Java, firstline=502, lastline=502, firstnumber=502]{../src/StandardService.java}

		    	  \lstinputlisting[language=Java, firstline=505, lastline=505, firstnumber=505]{../src/StandardService.java}

		    	  These comments don't explain what the code are doing.

		    \item \textbf{Checklist \hyperref[C:19]{C.19}} \\

		    	  \lstinputlisting[language=Java, firstline=475, lastline=475, firstnumber=475]{../src/StandardService.java}

		    	  \lstinputlisting[language=Java, firstline=492, lastline=498, firstnumber=492]{../src/StandardService.java}

		    	  \lstinputlisting[language=Java, firstline=503, lastline=504, firstnumber=503]{../src/StandardService.java}

		    	  Commented code does not contain a reason for being commented out.

		    \item \textbf{Checklist \hyperref[C:23]{C.23}} \\

		          \lstinputlisting[language=Java, firstline=476, lastline=476, firstnumber=476]{../src/StandardService.java}

		          No “@throws” javadoc field for the exception "LifecycleException"

		    \item \textbf{Checklist \hyperref[C:33]{C.33}} \\

		    	  \lstinputlisting[language=Java, firstline=506, lastline=507, firstnumber=506]{../src/StandardService.java}

		    	  Declarations are not at the top of the block.

		    \item \textbf{Checklist \hyperref[C:40]{C.40}} \\

		          \lstinputlisting[language=Java, firstline=482, lastline=482, firstnumber=482]{../src/StandardService.java}

		          Two objects are compared using "==" and not equals().

		    \item \textbf{Checklist \hyperref[C:44]{C.44}} \\

		          \lstinputlisting[language=Java, firstline=480, lastline=486, firstnumber=480]{../src/StandardService.java}

		          There is a "break;" into the for() block. Use another iteration block.

		          \lstinputlisting[language=Java, firstline=510, lastline=510, firstnumber=510]{../src/StandardService.java}

		          It's better to explicitly increment 'k' before the assignment with a separed statement.

		          These lines of code does not avoid "Brutish Programming".

		    \item \textbf{Checklist \hyperref[C:51]{C.51}} \\

		    	  \lstinputlisting[language=Java, firstline=515, lastline=515, firstnumber=515]{../src/StandardService.java}

		    	  The type of the second parameter has to be "Object" and "connector" is of the type "Connector".

		    \item \textbf{Checklist \hyperref[C:56]{C.56}} \\

		    	  \lstinputlisting[language=Java, firstline=480, lastline=486, firstnumber=480]{../src/StandardService.java}

		    	  Loops are not correctly formed expecially the termination expression at the line 484 (break).

		\end{itemize}

\end{document}