\documentclass[../../../../projectPlan.tex]{subfiles}

\begin{document}

	\section{Project Size}

		The purpose of this section is to estimate Function Points to give an estimation of the project size.
		We will use this Size Estimation Procedure:

		\begin{itemize}
			\item Determine the function counts by type: Count the number of functions for each Function Type.
			\item Determine the complexity level for each Function Type.
			\item Apply weights to the Function Types.
			\item Compute the Function Points for each Function Type.
		\end{itemize}
		
		Each subsection will take in account a different User Function Type. User Function Types are described in the following table:

		\begin{table}[H]
			\centering
			\label{my-label}
			\begin{tabular}{ll}
				\hline
				\textbf{External Input (Inputs)}                                                         & \begin{tabular}[c]{@{}l@{}}Count each unique user data or user control input \\ type that (i) enters the external boundary of the \\software system being measured and (ii) adds or changes\\data in a logical internal file.\end{tabular}                                                             \\ \hline
				\textbf{External Output (Outputs)}                                                       & \begin{tabular}[c]{@{}l@{}}Count each unique user data or control output type\\that leaves the external boundary of the software system\\being measured.\end{tabular}                                                                                                                                \\ \hline
				\begin{tabular}[c]{@{}l@{}}\textbf{Internal Logical File}\\ \textbf{(Files)}\end{tabular}         & \begin{tabular}[c]{@{}l@{}}Count each major logical group of user data or control\\information in the software system as a logical internal\\file type. Include each logical file (e.g., each logical group\\of data) that is generated, used, or maintained by the\\software system.\end{tabular} \\ \hline
				\begin{tabular}[c]{@{}l@{}}\textbf{External Interface Files}\\ \textbf{(Interfaces)}\end{tabular} & \begin{tabular}[c]{@{}l@{}}Files passed or shared between software systems should\\be counted as external interface file types within each system.\end{tabular}                                                                                                                                     \\ \hline
				\textbf{External Inquiry (Queries)}                                                      & \begin{tabular}[c]{@{}l@{}}Count each unique input-output combination, where an\\input causes and generates an immediate output, as an\\external inquiry type.\end{tabular}                                                                                                                          \\ \hline
			\end{tabular}
			\caption{Function Types}
		\end{table} 

		To determine the complexity level of each Function Type, it's used the following tables:

		\begin{table}[H]
			\centering
			\label{my-label}
				\begin{tabular}{cccc}
				\hline
				\multicolumn{4}{c}{\textbf{For ILF and EIF}}                                                                                                                 \\ \hline
				\multicolumn{1}{c|}{\multirow{2}{*}{\textbf{\begin{tabular}[c]{@{}c@{}}Record\\ Elements\end{tabular}}}} & \multicolumn{3}{c|}{\textbf{Data Elements}}       \\
				\multicolumn{1}{c|}{}                                                                                    & \textbf{1 - 19} & \textbf{20 - 50} & \textbf{51+} \\ \hline
				\multicolumn{1}{c|}{1}                                                                                   & Low             & Low              & Average      \\ \hline
				\multicolumn{1}{c|}{2 - 5}                                                                               & Low             & Average          & High         \\ \hline
				\multicolumn{1}{c|}{6+}                                                                                  & Average         & High             & High         \\ \hline
			\end{tabular}
			\caption{External Inputs and External Interface Files complexity distribution}
		\end{table}


		\begin{table}[H]
		\centering
		\label{my-label}
		\begin{tabular}{cccc}
			\hline
			\multicolumn{4}{c}{\textbf{For EO and EQ}}                                                                                                                 \\ \hline
			\multicolumn{1}{c|}{\multirow{2}{*}{\textbf{\begin{tabular}[c]{@{}c@{}}Record\\ Elements\end{tabular}}}} & \multicolumn{3}{c|}{\textbf{Data Elements}}     \\
			\multicolumn{1}{c|}{}                                                                                    & \textbf{1 - 5} & \textbf{6 - 19} & \textbf{20+} \\ \hline
			\multicolumn{1}{c|}{0 or 1}                                                                              & Low            & Low             & Average      \\ \hline
			\multicolumn{1}{c|}{2 - 3}                                                                               & Low            & Average         & High         \\ \hline
			\multicolumn{1}{c|}{4+}                                                                                  & Average        & High            & High         \\ \hline
		\end{tabular}
		\caption{External Output and External Inquiries complexity distribution}
		\end{table}


		\begin{table}[H]
		\centering
		\label{my-label}
		\begin{tabular}{cccc}
			\hline
			\multicolumn{4}{c}{\textbf{For EI}}                                                                                                                        \\ \hline
			\multicolumn{1}{c|}{\multirow{2}{*}{\textbf{\begin{tabular}[c]{@{}c@{}}Record\\ Elements\end{tabular}}}} & \multicolumn{3}{c|}{\textbf{Data Elements}}     \\
			\multicolumn{1}{c|}{}                                                                                    & \textbf{1 - 4} & \textbf{5 - 15} & \textbf{16+} \\ \hline
			\multicolumn{1}{c|}{0 or 1}                                                                              & Low            & Low             & Average      \\ \hline
			\multicolumn{1}{c|}{2 - 3}                                                                               & Low            & Average         & High         \\ \hline
			\multicolumn{1}{c|}{3+}                                                                                  & Average        & High            & High         \\ \hline
		\end{tabular}
		\caption{External Inputs complexity distribution}
		\end{table}

		To determine the weights for each Function type, the following table has been used (for each Function Type, a weight is assigned):


		\begin{table}[H]
		\centering
		\label{my-label}
		\begin{tabular}{cccc}
			\hline
			\multicolumn{4}{c}{\textbf{Weights}}                                                                           \\ \hline
			\multicolumn{1}{c|}{\multirow{2}{*}{\textbf{Function Type}}} & \multicolumn{3}{c|}{\textbf{Complexity-Weight}} \\
			\multicolumn{1}{c|}{}                                        & \textbf{Low} & \textbf{Average} & \textbf{High} \\ \hline
			\multicolumn{1}{c|}{Internal Logical Files}                  & 7            & 10               & 15            \\ \hline
			\multicolumn{1}{c|}{External Interface Files}                & 5            & 7                & 10            \\ \hline
			\multicolumn{1}{c|}{External Inputs}                         & 3            & 4                & 6             \\ \hline
			\multicolumn{1}{c|}{External Outputs}                        & 4            & 5                & 7             \\ \hline
			\multicolumn{1}{c|}{External Inquiries}                      & 3            & 4                & 6             \\ \hline
		\end{tabular}
		\caption{Function Types Weights}
		\end{table}

		\subsection{Internal Logical Files}

			Internal Logical Files are stored for:
			\begin{itemize}
				\item \textbf{Users (Customers and Taxi Drivers)} \\
				      Users have to memorize from 8 to 5 elements and for each user there are two external records that have to be memorized (Position and Zone). So, according to the table 2.2, the complexity of "Users" is LOW and, according to the table 2.5 the Functional Points assigned are 7.

				\item \textbf{Rides (Reservations and Single Rides)} \\
				      Rides have to memorize from 3 to 1 elements and for each ride there are three external records that have to be memorized (Position, Customer and Taxi Driver). So, according to the table 2.2, the complexity of "Rides" is LOW and, according to the table 2.5 the Functional Points assigned are 7.

				\item \textbf{City Zones} \\
				      City Zones have to memorize from 3 to 5 elements and for each user there are one external record that have to be memorized. So, according to the table 2.2, the complexity of "City Zones" is LOW and, according to the table 2.5 the Functional Points assigned are 7.
			\end{itemize}

			The computation of the final Functional Points assigned to the class "Internal Logical Files" is:\\
			FP(ILF) = 7+7+7 = 21.


		\subsection{External Interface Files}
			External Interface Files are stored for:
			\begin{itemize}
				\item \textbf{GPS} \\
					   Informations about the GPS position of the users are essentialy represented as a "Position" (Latitude and Longitude). So according to the table 2.2 the complexity is LOW and according to the table 2.5 the Functional Points assigned are 5.

			    \item \textbf{Facebook Accounts}\\
			          Informations about the users that comes from the Facebook Login System, are represented as Customers (5 attributes). The external records are two (Position and Zone). So According to the table 2.2 the complexity is LOW and according to the table 2.5 the Functional Points assigned are 5. 
			\end{itemize}

			The computation of the final Functional Points assigned to the class "External Interface Files" is:\\
			FP(EIF) = 5+5 = 10.

		\subsection{External Inputs}
			External Inputs are:
			\begin{itemize}
				\item \textbf{Registration (of a Customer, with Facebook and of a Taxi Driver)}\\
				      These operations involve only one data type (User) and involve less than 5 elements.
				      According to the table 2.4 the complexity is LOW and the Functional Points assigned to each operation is 3.
				      The overall Functional Points assigned are: FP(Registration) = 3 x 3 = 9.

				\item \textbf{Login/Logout} \\
				      These operations involve only one data type (User) and involve less than 5 elements.
				      According to the table 2.4 the complexity is LOW and the Functional Points assigned to each operation is 3.
				      The overall Functional Points assigned are: FP(Login/Logout) = 3 x 2 = 6.

				\item \textbf{Profile Modification (Availability or Details modification)} \\
				      These operations involve only one data type (User) and involve less than 5 elements.
				      According to the table 2.4 the complexity is LOW and the Functional Points assigned to each operation is 3.
				      The overall Functional Points assigned are: FP(Profile Modification) = 3 x 2 = 6.

				\item \textbf{Report Abuse (Customer or Taxi Driver)} \\
                      These operations involve only one data type (User) and involve less than 5 elements.
				      According to the table 2.4 the complexity is LOW and the Functional Points assigned to each operation is 3.
				      The overall Functional Points assigned are: FP(Report Abuse) = 3 x 2 = 6.

				\item \textbf{Taxi Request}
				      These operations involve only one data type (Ride) and involve less than 15 elements.
				      According to the table 2.4 the complexity is LOW and the Functional Points assigned to each operation is 3.
				      The overall Functional Points assigned are: FP(Taxi Request) = 3 x 1 = 3.

				\item \textbf{Taxi Reservation/Reservation Deletion}
				      These operations involve only one data type (Ride) and involve less than 15 elements.
				      According to the table 2.4 the complexity is LOW and the Functional Points assigned to each operation is 3.
				      The overall Functional Points assigned are: FP(Report Abuse) = 3 x 2 = 6.

				\item \textbf{Accept/Decline Ride}
				      These operations involve two data types (Taxi Driver (priority) and Ride) and involve less than 15 elements.
				      According to the table 2.4 the complexity is AVERAGE and the Functional Points assigned to each operation is 3.
				      The overall Functional Points assigned are: FP(Accept/Decline Ride) = 7 x 1 = 7.
			\end{itemize}
			The total Functional Points assigned to the class "External Inputs" are: \\
			FP(EI) = FP(Accept/Decline Ride) + FP(Report Abuse) + FP(Taxi Request) + FP(Report Abuse) + FP(Profile Modification) + FP(Login/Logout) + FP(Registration) = 43

		\subsection{External Inquiries}
			Functionalities that causes External Inquiries are:
			\begin{itemize}
				\item \textbf{View Reservations}
				      This functionality involve only one data type (Rides) and can potentially involve more than 20 elements. According to the table 2.3 the complexity is AVERAGE and according to the table 2.5 the Functional Points assigned are 4.

				\item \textbf{View Profile Informations}
				      This functionality involve only one data type (User) and can involve from 5 to 8 elements. According to the table 2.3 the complexity is LOW and according to the table 2.5 the Functional Points assigned are 3.
			\end{itemize}

			The computation of the final Functional Points assigned to the class "External Inquiries" is:\\
			FP(EI) = 4+3 = 7.

		\subsection{External Outputs}
		    The operations that generate External Outputs are:
		    \begin{itemize}
		    	\item \textbf{Ride Request Notification} \\
		    	      For this operation are involved two data types (Zones, Users) and are considered from 6 to 19 elements. According to the table 2.3 the complexity is AVERAGE and according to the table 2.5 the Functional Points assigned are 5.

		    	\item \textbf{Incoming Taxi Notification} \\
		              For this operation are involved two data types (Ride Request, Users) and are considered from 6 to 19 elements. According to the table 2.3 the complexity is AVERAGE and according to the table 2.5 the Functional Points assigned are 5.
		    \end{itemize}

		    The computation of the final Functional Points assigned to the class "External Outputs" is:\\
			FP(EO) = 5+5 = 10.

		\subsection{Computation of Total Functional Points}

			The final computation of the Functional Points is: \\
			FP = FP(ILF) + FP(EIF) + FP(EInputs) + FP(EInquiries) + FP(EO) = 21 + 10 + 43 + 7 + 10 = 91.


			For Line of Code count, we consider two main programming languages: Java and C++.
			The computation of the Lines of Code in functions of the Functional Points is:
			\begin{itemize}
				\item \textbf{Java} (SLOC/FP = 53) is: TotalSLOC = 91 x 53 = 4823.
				\item \textbf{C++} (SLOC/FP = 55) is: TotalSLOC = 91 x 55 = 5005.
			\end{itemize} 


\end{document}