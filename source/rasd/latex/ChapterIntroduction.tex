\newpage
\chapter{Introduction}

	\section{Purpose}
	This document is the R.A.S.D. (Requirement Analysis and Specification Document).
	The purpose of this document is the description of the "myTaxiService" system. 
	At first, it will provide functional and non-functional requirements, a complete overview of the constraints of the system and its limits. Then it will explain in detail the dynamics of the system using real-life use cases.
	Finally this document will provide a base for the developers that concretely have to implement the system.

	\section{Actual System}
	The functionality that the new system will provide is now not supported. 
	So the entire system must be developed without using or modifying existing system.

	\section{Scope}
	The objective of “myTaxiService” is to provide an interface between \hyperref[sec:customer]{customers} and \hyperref[sec:tdriver]{taxi drivers} to optimize their interaction and provide a fair management of taxi queues. The \hyperref[sec:normaluser]{users}, once registered through the mobile application or the web application, can request a taxi for their travel or reserve one, specifying the origin and the destination. The reservation can be done at least two hour before the ride; if the reservation can take place, the system will allocate a taxi 10 minutes before the meeting time.
	On the other side, \hyperref[sec:tdriver]{taxi drivers} can inform the system that they are waiting for a client and accept or decline a ride request. If the request has been accepted, a notification will be sent to the requesting \hyperref[sec:customer]{customer} with the identification number of the incoming taxi and the time he has to wait. Otherwise, if the request has been rejected it will be forwarded to the next taxi in the queue.
	The system has to optimize the management of \hyperref[sec:customer]{customers} requests giving the rides to the taxi with the highest priority that has to be evaluated in function of avaiability and the nearness of the \hyperref[sec:tdriver]{taxi driver}.

	\section{Actors}
		\begin{itemize}
		  \item \textbf{Guest User:}\label{sec:normaluser} guest users are unlogged or unregistered users. They can visit the login page or the registration forms.

		  \item \textbf{Registered User:}\label{sec:ruser} this kind of user identify either a Guest User or a Taxi Driver.

		  \item \textbf{Customer:}\label{sec:customer} this kind of user is the end-user of the service. He can perform request for taxis or reserve a ride. In his personal page he can view his requests and the system responses.

		  \item \textbf{Taxi Driver:}\label{sec:tdriver} this kind of user is composed by the actual taxi drivers that can only see customers requests that has been forwarded by the system. He can accept or decline these requests. Also, he's considered a special kind of user because one can register as a "Taxi Driver" only if he provide a valid Taxi licence.
		\end{itemize}

	\section{Goals}

		\begin{itemize}
			\item \textbf{\lbrack G.1\rbrack}\label{sec:g1} Allow \hyperref[sec:normaluser]{guest user} to become a \hyperref[sec:customer]{customer} creating a myTaxiService Account.

			\item \textbf{\lbrack G.2\rbrack}\label{sec:g2} Allow \hyperref[sec:normaluser]{guest user} to become a \hyperref[sec:customer]{customer} using his Facebook Account.

			\item \textbf{\lbrack G.3\rbrack}\label{sec:g3} Allow \hyperref[sec:normaluser]{guest user} to become a \hyperref[sec:tdriver]{taxi driver}.

			\item \textbf{\lbrack G.4\rbrack}\label{sec:g4} Allow \hyperref[sec:ruser]{registered user} to log in with myTaxiService account.

			\item \textbf{\lbrack G.5\rbrack}\label{sec:g5} Allow \hyperref[sec:customer]{customer} to log in with Facebook account.

			\item \textbf{\lbrack G.6\rbrack}\label{sec:g6} Allow a \hyperref[sec:normaluser]{Registered User} to view or modify his username and email.

			\item \textbf{\lbrack G.7\rbrack}\label{sec:g7} Allow a \hyperref[sec:normaluser]{Registered User} to retrieve his password if he doesn't remember it.

			\item \textbf{\lbrack G.8\rbrack}\label{sec:g8} Allow a \hyperref[sec:normaluser]{RegisteredUser} to signal another one if he has made a bad use of the system.

			\item \textbf{\lbrack G.9\rbrack}\label{sec:g9} Allow \hyperref[sec:customer]{customers} to require a taxi.

			\item \textbf{\lbrack G.10\rbrack}\label{sec:g10} Allow \hyperref[sec:customer]{customers} to reserve a ride.

			\item \textbf{\lbrack G.11\rbrack}\label{sec:g11} Allow \hyperref[sec:customer]{customers} to delete a previous reservations.

			\item \textbf{\lbrack G.12\rbrack}\label{sec:g12} Allow \hyperref[sec:tdriver]{taxi drivers} to accept or decline a ride request.

			\item \textbf{\lbrack G.13\rbrack}\label{sec:g13} Allow \hyperref[sec:tdriver]{taxi drivers} to notify their availability.
		\end{itemize}
		
	\section{Definitions, Acronyms, Abbreviations}
		
		\subsection{Definitions}
			\begin{itemize} 
			
				\item Requesting User: user that requests a taxi ride. This ride can be a Reservation or a Normal ride.
				
				\item Reservation (or "Reserve a Ride") : A ride that is reserved for a future journey, the taxi is not called immediately 				but will be called at the time decided by the user.
				
				\item Normal Ride: A ride that is call to be handled as soon as possible. If is not specified, a ride is considered as a 						normal ride.
				
				\item Zone: We decide to divide the city area into zone. this organization is done to optimize the service. The shape 						and the dimension of this zone is not decided yet.
				
				\item Zone Queue: This Queue is composed of taxi drivers. With this queue we can organize the taxi, in order of 						priority, to optimize the service. Each Zone will have exactly one Queue of TaxiDrivers.

		\subsection{Acronyms}
			\begin{itemize}
				\item RASD: Requirement Analysis and Specification Documents.
				\item DD: Design Document.
				\item UML: Unified Modeling Language.
				\item OS: Operative System.
				\item API: Application Program Interface.
				\item GPS: Global Positioning System.
				\item HTTP: Hypertext Transfer Protocol.
				\item HTTPS: Secure Hypertext Transfer Protocol.
			\end{itemize}

		\subsection{Abbreviations}
			\begin{itemize}
				\item G.x is the x-Goal.
				\item Req.x is the x-Functional Requirement.
				\item Dom.x is the x-Domain Assumption.
				\item S.x is the x-Scenario.
			\end{itemize}
		
	\section{Reference documents}
		\begin{itemize}
			\item \href{http://www.math.uaa.alaska.edu/~afkjm/cs401/IEEE830.pdf}{(IEE830) IEEE Recommended Practice for Software Requirements Specifications}
		\end{itemize}
		
	\section{Document overview.}
	Until now, we have given a general explanation about the software functionalities and a brief description of this document. Now we will describe what the rest of this RASD contains.\\
	In Section 2 we will focus more on system constraints and assumptions.\\
	In Section 3 we will describe requirements, typical scenarios and use-cases. In this section there is also a collection of UML diagrams that describes in particular the functionalities of the system.\\
	In Section 4 we will describe the alloy documentation about our system and describe all the software used to make this document.