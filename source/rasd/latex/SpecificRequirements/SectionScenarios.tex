\section{Scenarios}
	Here are described in natural language some useful scenarios that shows how the system should work in different cases:

		\subsection{S.1 Registration as a Customer creating a myTaxiService account}\label{sec:NormalCustomerRegistrationScenario}
		To read the Functional Requirements for this scenario, refer to \hyperref[sec:frs1]{G.1 Functional Requirements Specification}\\

		Aristotle is new in the system and he wants to simplify the way he calls for a taxi. So he download the "myTaxiService" application from the Play Store (since he has a smartphone with the fancy Android 5.0 Lollipop Operative System) and he open it.
		The application shows to Aristotle the login screen and he taps on the "Sign Up as a Customer" button (see \hyperref[login_m]{Login Screen}) going to the registration page. In that page, he has to fill the form (see \hyperref[cregistration_m]{Customer Registration Screen (Mobile)}) and after that he clicks on the "Accept" button. If the data entered by Aristotle are ok, he is automatically logged in and he can start using myTaxiService. Otherwise, he is redirected to an error page, that describes what is the problem. 

		\subsection{S.2 Registration as a Customer using a Facebook Account}\label{sec:FacebookCustomerRegistrationScenario}
		To read the Functional Requirements for this scenario, refer to \hyperref[sec:frs2]{G.2 Functional Requirements Specifications}\\

		Pythagoras is new in the system and he wants to simplify the way he calls for a taxi, but he doesn't want to loose time inserting his credential into a boring registration form, so he decide to register with his Facebook Account.
		He download the application from the Apple App Store (since he has a glorious iPhone 6 Plus) and he open it.
		The application shows to Pythagoras the login screen and he taps on the "Log in With Facebook" button calling the Facebook Login API.
		If the registration is successful, the Facebook System will reply with the informations about the user and the myTaxiService system can create a record in the database with Pythagoras informations. Otherwise, he is redirected to an error page, that describes what is the problem.

		\subsection{S.3 Registration as a Taxi Driver creating a new Taxi Driver myTaxiService account.}\label{sec:TaxiDriverRegistrationScenario}
		To read the Functional Requirements for this scenario, refer to \hyperref[sec:frs3]{G.3 Functional Requirements Specifications}\\

		Thales is a great taxi worker and he wants to improve his service to the customers so he decide to join myTaxiService.
		First of all he visit the myTaxiService web page from his pc. That page shows the login screen (see \hyperref[login_m]{Login Screen}) and he clicks on the "Sign up as a Taxi Driver" button. The web application redirect him to the Taxi Drivers registration page. Thales fills up the form (see \hyperref[tregistration_w]{Taxi Driver Registration Screen (Web)}) and clicks on the "Accept" button. The system perform a special check for the validity of the given taxi licence and taxi number. If the data entered by Thales are ok, he is automatically logged in and he can start using myTaxiService as a Taxi Driver. Otherwise, he is redirected to an error page, that describes what is the problem.

		\subsection{S.4 Login with a myTaxiService account.}\label{sec:RegisteredUserLoginScenario}
		To read the Functional Requirements for this scenario, refer to \hyperref[sec:frs4]{G.4 Functional Requirements Specifications}\\

		Epicurus is a Registered User and he wants to login to use the myTaxiServer Application with his smartphone. So he taps on the application icon and start using it. myTaxiService Application shows to him the login screen and Epicurus fills up the form with his credentials. After that he submits the data by pressing the "Login" button. The Application calls the server function to check the credentials of Epicurus and if them are ok, the server reply to the application with a success message and the homepage is shown to Epicurus. Otherwise the server reply with an error message that is shown to Epicurus.
