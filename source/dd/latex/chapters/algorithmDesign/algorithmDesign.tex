\documentclass[../../dd.tex]{subfiles}

\begin{document}

	\chapter{Algorithm Design}
		This section is intended to better explain some functions and algorithms that can be ambiguous in the previous presentation of our system.


		Algoritmi speciali usati nel nostro sistema, uno per uno descritti con pseudocodice o flow chart

		algoritmi che potremmo mettere:
		\begin{itemize}
				\item quello per selezionare il messaggio e gestirlo (IO Manager)

					1)procedure selectMessage()
						
				\item taxi allocator (selezione delle ride non gestite e gestione)
				\item queues manager (check positions e ordinamento delle code)
				\item modify priority --> recall della funzione di ordinamento del queue manager
				\item credentials checker (check email, password)
		\end{itemize}

		\begin{algorithm}
			\caption{Descrizione Algoritmo}\label{alg:descrizioneAlgoritmo}
			\begin{algorithmic}[1]
			\Procedure{Example}{$par1,par2$}\Comment{example algorithm}
				\State $par1\gets 1$

				\If {$i\geq maxval$}
				    \State $i\gets 0$
				\Else
				    \If {$i+k\leq maxval$}
				        \State $i\gets i+k$
				    \EndIf
				\EndIf
			\EndProcedure
			\end{algorithmic}
		\end{algorithm}

		\section{IO Manager Algorithms}
			Here we describe some functions and algorithms relatives to the IO Manager component.
			\subsection{Select Message}
				The select message procedure have to select the right message handler for an incoming/outgoing message.
				It's described by the following pseudocode:

				\begin{algorithm}
					\caption{Select Message Procedure}\label{alg:selectMessage}
					\begin{algorithmic}[1]
					\Procedure{SelectMessage}{$rawMessage$}\Comment{rawMessage is an xml (or other markup language like json) message}
						\State $messageHandler\gets messagesTypeMap.getMessageHandler($
						\State $rawMessage.type)$
						\State $return messageHandler.handleMessage(rawMessage)$
					\EndProcedure
					\end{algorithmic}
				\end{algorithm}

				\textbf{Note on handleMessage():} every single message class that is present on the IO Manager component, can handle incoming messages or outgoing messages and the implementation of handleMessage() for each of this sub-components is different depending on the message that has to be handled.

		\section{Requests and Reservations Manager Algorithms}
			Here we describe some functions and algorithms relatives to the Requests and Reservations Manager component.

			\subsection{Check Rides To Allocate}
				The checkRidesToAllocate procedure is invoked by a scheduler every 2 minutes.
				This procedure has to check if there are rides or reservations that has to be handled and to put them into the ride-handling stack.

				\begin{algorithm}
					\caption{Check Rides to Allocate}\label{alg:checkRidesToAllocate}
					\begin{algorithmic}[1]
					\Procedure{checkRidesToAllocate}{}
						\State $tempList\gets DatabaseManager.getReady()$\Comment{get all the rides that need to be handled}
						\ForAll{$element$ in $tempList$}
							\State $allocateRide(element)$\Comment{handle the single ride}
						\EndFor
					\EndProcedure
					\end{algorithmic}
				\end{algorithm}

			\subsection{Allocate Ride}
				Allocate Ride starts a new Thread to perform the handling in parallel with the others operations.

				\begin{algorithm}
					\caption{Allocate Ride}\label{alg:allocateRide}
					\begin{algorithmic}[1]
					\Procedure{allocateRide}{$ride$}
						\State $taxiID\gets 0$\Comment{initialization}
						\Repeat
							\State $taxiID\gets ZonesManager.getFirstAvailableTaxiDriver()$
							\State $msgTD\gets createMessage("TaxiDriverNotif", taxiID)$
							\State $response\gets IOManager.selectMessage(msgTD)$
							\State $ZonesManager.changePriority(0, taxiID)$
						\Until{$response == decline$}
						\State $DatabaseManager.setAvailability(taxiID, false)$
						\State $msgC\gets createMessage("CustomerNotif", ride.CustomerID)$
						\State $IOManager.selectMessage(msgC)$
					\EndProcedure
					\end{algorithmic}
				\end{algorithm}

		\section{Zones Manager Algorithms}
			Here we describe some functions and algorithms relatives to the Zones Manager component.

			\subsection{Check Positions}
				This procedure is invoked by a scheduler every 2 minutes and updates the zones  of every Taxi Driver.

				\begin{algorithm}
					\caption{Check Positions}\label{alg:checkPosition}
					\begin{algorithmic}[1]
					\Procedure{checkPositions}{}
						\State $availableDrivers\gets DatabaseManager.getAvailableTDs()$
						\ForAll{$dirver$ in $availableDrivers$}
							\State $driverZone\gets ZoneCalculator.calculateZone($
							\State $driver.position)$
							\State $driver.setZone(driverZone)$
						\EndFor
						\State $orderQueues(availableDrivers)$
					\EndProcedure
					\end{algorithmic}
				\end{algorithm}

			\subsection{Order Queues}
				This procedure has to order the queues in function of the priority of the Taxi Drivers.

				\begin{algorithm}
					\caption{Order Queues}\label{alg:orderQueues}
					\begin{algorithmic}[1]
					\Procedure{orderQueues}{$drivers$}
						\State $QueuesManager.resetQueues()$
						\State $setCurrentDrivers(drivers)$
						\ForAll{$driver$ in $drivers$}
							\State $zoneQueue\gets QueuesManager.getQueue(driver.zone)$
							\State $zoneQueue.orderedInsert(driver)$\Comment{insert the driver ordering by the priority}
						\EndFor
					\EndProcedure
					\end{algorithmic}
				\end{algorithm}

			\subsection{Modify Priority}
				This procedure has to modify the priorities of the current queued Taxi Drivers and reinsert the driver in his queue with ordered

				\begin{algorithm}
					\caption{Modify Priority}\label{alg:modifyPriority}
					\begin{algorithmic}[1]
					\Procedure{modifyPriority}{$new, taxiID$}
						\State $driver\gets getFromCurrentDrivers(taxiID)$
						\State $driverQueue\gets QueueManager.getQueue(driver.zone)$
						\State $driverQueue.remove(taxiID)$\Comment{remove from the current zone}
						\State $driver.setPriority(new)$
						\State $driverQueue.orderedInsert(driver)$\Comment{reinsert in order}
					\EndProcedure
					\end{algorithmic}
				\end{algorithm}

		\section{Profile Manager Algorithms}
			Here we describe some functions and algorithms relatives to the Profile Manager component.

			\subsection{Check Email}
				This procedure is in charge to tell if the given email is a valid one.

				\begin{algorithm}
					\caption{Check Email}\label{alg:checkEmail}
					\begin{algorithmic}[1]
					\Procedure{checkEmail}{$emaik$}
						\State $regex\gets "[a-zA-Z1-9]+@[a-zA-Z ]?.[a-zA-Z]{2,3}"$
						\If{$regex.match(email)$}
							\State return $true$
						\EndIf
						\State return $false$
					\EndProcedure
					\end{algorithmic}
				\end{algorithm}

			\subsection{Check Password}
				This procedure is in charge to tell if the given password is a valid one.

				\begin{algorithm}
					\caption{Check Password}\label{alg:checkPassword}
					\begin{algorithmic}[1]
					\Procedure{checkPassword}{$password$}
						\State $length\gets password.length()$
						\State $numbers\gets containsDigits(password)$
						\State $letters\gets containsCharacters(password)$
						\State return $((length>8)$ and $numbers$ and $letters)$
					\EndProcedure
					\end{algorithmic}
				\end{algorithm}

			\subsection{Check Username}
				This procedure is in charge to tell if the given username is a valid one.

				\begin{algorithm}
					\caption{Check Username}\label{alg:checkUsername}
					\begin{algorithmic}[1]
					\Procedure{checkUsername}{$username$}
						\State return $(username.length()>8)$
					\EndProcedure
					\end{algorithmic}
				\end{algorithm}

\end{document}