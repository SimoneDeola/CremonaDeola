\documentclass[../../../../dd.tex]{subfiles}

\begin{document}

	\section{Selected Architectural Styles and Patterns}
		\subsection{Patterns}
		\begin{itemize}
		\item The main pattern we use is the \textbf{MVC pattern}. We choose this pattern because of all the advantages it involves. For example the separation between the three main layer makes the system compatible with possible future changing. The fact that the View is divided from the rest of the system gives the possibility to implement different kind of UI, that is required for this system. Also, the separation of the three part of the system make possible a Parallel development that speed-up the creation of the system. And so on.
		
		\item The previous pattern is implemented on a \textbf{three-tier architecture}. This architecture splits the tre part of MVC on three different tier. The Model part is implemented on the Data Tier, the Control part is implemented in the Logic Tier and the view part is implemented in the Presentation Tier. This pattern is the best choice if you choose to implement a MVC pattern.
		
		\item The IO Manager, implement a \textbf{Command Pattern}. We choose this pattern to handle the command sent to the server because it makes the communication simpler and make the system scalable. For example if we want to implement a new command that the user can do, we just add a new command class and make it handable for the Message handler class. At this point the previous system still working and if the user want to use the new functionality should only update his front end application.
		
		
		\item The Ride class implement the \textbf{State Pattern}. We choose this pattern because of the nature of the class that change different state in his lifecycle.
		
		\end{itemize}
		

	
\end{document}